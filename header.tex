\usepackage[pdftex]{color}

\usepackage{rotating}

\usepackage[margin=0pt,font=small,labelfont=bf]{caption}

\usepackage{tikz}
\usepackage{pgfplots}
\pgfplotsset{compat=1.15}
\usepackage{varwidth}
\usepackage{xspace}
\usetikzlibrary{matrix, arrows, calc, positioning, decorations, arrows.meta, fit, shapes.geometric,decorations.pathreplacing}

\pgfdeclaredecoration{sl}{initial}{
	\state{initial}[width=\pgfdecoratedpathlength-1sp]{
		\pgfmoveto{\pgfpointorigin}
	}
	\state{final}{
		\pgflineto{\pgfpointorigin}
	}
}

\definecolor{p1}{named}{blue}
\definecolor{p2}{named}{red}
\definecolor{p3}{rgb}{0.0,0.8,0.1}
\definecolor{p4}{rgb}{0.8,0.8,0.0}
\definecolor{p5}{named}{orange}
\definecolor{p6}{named}{cyan}
\definecolor{p7}{named}{magenta}
\definecolor{p8}{named}{gray}

\definecolor{per1}{rgb}{0.0, 0.0, 1.0}
\definecolor{per2}{rgb}{0.0, 0.5, 1.0}
\definecolor{per3}{rgb}{0.19, 0.55, 0.91}
\definecolor{per4}{rgb}{0.11, 0.67, 0.84}
\definecolor{per5}{rgb}{0.36, 0.54, 0.66}
\definecolor{per6}{rgb}{0.0, 0.81, 0.82}
\definecolor{per7}{rgb}{0.03, 0.91, 0.87}
\definecolor{per8}{rgb}{0.0, 0.8, 0.6}
\definecolor{per9}{rgb}{0.67, 0.88, 0.69}
\definecolor{per10}{rgb}{0.52, 0.73, 0.4}
\definecolor{per11}{rgb}{0.5, 1.0, 0.0}
\definecolor{per12}{rgb}{0.8, 1.0, 0.0}
\definecolor{per13}{rgb}{0.6, 0.98, 0.6}
\definecolor{per14}{rgb}{0.99, 0.97, 0.37}
\definecolor{per15}{rgb}{0.98, 0.85, 0.37}
\definecolor{per16}{rgb}{1.0, 0.62, 0.0}
\definecolor{per17}{rgb}{0.93, 0.51, 0.93}
\definecolor{per18}{rgb}{0.9, 0.4, 0.38}
\definecolor{per19}{rgb}{1.0, 0.71, 0.76}
\definecolor{per20}{rgb}{0.99, 0.74, 0.71}
\definecolor{per21}{rgb}{0.89, 0.02, 0.17}
\definecolor{per22}{rgb}{1.0, 0.27, 0.0}
\definecolor{per23}{rgb}{0.68, 0.05, 0.0}
\definecolor{per24}{rgb}{0.51, 0.41, 0.33}
\definecolor{per25}{rgb}{0.85, 0.54, 0.4}

\definecolor{match}{rgb}{0.99,0.78,0.07}
\definecolor{bcp}{named}{blue}
\definecolor{pcp}{rgb}{0.17,0.78,0.15}

\tikzset{parallel arrow/.style={latex-,pcp,
		shorten >=0mm, shorten <=0mm, 
		decoration={sl,raise=1mm},decorate}}
\tikzset{parallel arrow1/.style={latex-,pcp,
		shorten >=0mm, shorten <=0mm, 
		decoration={sl,raise=2mm},decorate}}
\tikzset{parallel arrow2/.style={latex-,dashed,bcp,
		shorten >=0mm, shorten <=0mm, 
		decoration={sl,raise=1mm},decorate}}
\tikzset{parallel arrow3/.style={latex-,dashed,bcp,
		shorten >=0mm, shorten <=0mm, 
		decoration={sl,raise=2mm},decorate}}
\tikzset{% \tikzstyle is deprecated
	block1/.style={ultra thick,color=blue},
	block2/.style={ultra thick,color=green},
	block3/.style={ultra thick,color=red},
	block4/.style={ultra thick,color=yellow},
	adjnode/.style={circle, draw=black, fill=white, align=center, thick},
	parent1/.style={pcp},
	parent2/.style={bcp},
	matchbegin/.style={},
	mismatch/.style={red!90!white},
	lce1/.style={circle, inner sep=0mm, minimum size=5mm, fill=pcp,text=white},
	lce2/.style={fill=bcp,text=white},
	r/.style={dashed}
}
\tikzset{% \tikzstyle is deprecated
	parrec/.style={blue,every mark/.append style={fill=blue!80!black},mark=*},
	parlp/.style={red,every mark/.append style={fill=red!80!black},mark=square*},
	parrnd/.style={brown!60!black,every mark/.append style={fill=brown!80!black},mark=otimes*},
	parls/.style={black,mark=star},
	pargr/.style={green,every mark/.append style={fill=green!80!black},mark=diamond*},
	parrndk/.style={orange,densely dashed,every mark/.append style={solid,fill=orange!80!black},mark=*},
	parrnddirk/.style={yellow!60!black,densely dashed,every mark/.append style={solid,fill=yellow!90!black},mark=square*}
}

\pgfplotscreateplotcyclelist{mycolorlist}{%
	blue,every mark/.append style={fill=blue!80!black},mark=*\\%
	red,every mark/.append style={fill=red!80!black},mark=square*\\%
	brown!60!black,every mark/.append style={fill=brown!80!black},mark=otimes*\\%
	black,mark=star\\%
	green,every mark/.append style={fill=green!80!black},mark=diamond*\\%
	orange,densely dashed,every mark/.append style={solid,fill=orange!80!black},mark=*\\%
	yellow!60!black,densely dashed,every mark/.append style={solid,fill=yellow!90!black},mark=square*\\%
	black,densely dashed,every mark/.append style={solid,fill=gray},mark=otimes*\\%
	blue,densely dashed,every mark/.append style={fill=blue!80!black},mark=star\\%
	red,densely dashed,every mark/.append style={solid,fill=red!80!black},mark=diamond*\\%
	green,densely dashed,every mark/.append style={solid,fill=green!80!black},mark=*\\%
}

\pgfplotsset{
	width=150mm,height=100mm,
	major grid style={thin,dotted,color=black!50},
	minor grid style={thin,dotted,color=black!50},
	grid,
	every axis/.append style={
		line width=0.5pt,
		tick style={
			line cap=round,
			thin,
			major tick length=4pt,
			minor tick length=2pt,
		},
	},
	legend cell align=left,
}

% Math definitions
\DeclareMathOperator*{\argmin}{argmin}
\DeclareMathOperator*{\argmax}{argmax}
\newcommand{\Ovon}[1]{\ensuremath{\mathcal{O}\left(#1\right)}}
\newcommand{\ovon}[1]{\ensuremath{o\left(#1\right)}}
\newcommand{\Omegavon}[1]{\ensuremath{\Omega\left(#1\right)}}
\newcommand{\Thetavon}[1]{\ensuremath{\Theta\left(#1\right)}}
\newcommand{\card}[1]{\ensuremath{\lvert #1\rvert}}
\newcommand{\set}[1]{\ensuremath{\left\lbrace #1\right\rbrace}}
\newcommand{\interval}[2]{\ensuremath{\lbrack#1..#2\rbrack}}
\newcommand{\abs}[1]{\card{#1}}
\newcommand{\idx}[2]{\ensuremath{#1\lbrack#2\rbrack}}
\newcommand{\euclid}[1]{\ensuremath{\parallel\nolinebreak[4]#1\nolinebreak[4]\parallel}}
\newcommand{\scalar}[2]{\ensuremath{\left\langle#1,#2\right\rangle}}

\newcommand{\lcespace}{$\Ovon{z\lg\left(\frac{N}{z}\right)}$\@\xspace}
\newcommand{\lcetime}{$\Ovon{\lg\left(N\right)}$\@\xspace}
\newcommand{\ttogtime}{$\Ovon{N}$\@\xspace}
\newcommand{\ttogspace}{$\Ovon{N}$\@\xspace}
\newcommand{\gtogtime}{$\Ovon{n\lg\left(\frac{N}{n}\right)}$\@\xspace}
\newcommand{\gtogspace}{$\Ovon{n+z\lg\left(\frac{N}{z}\right)}$\@\xspace}
\newcommand{\wordsize}{$\Omegavon{\lg\left(N\right)}$\@\xspace}
\newcommand{\stringsize}{$\Ovon{\card{w}}$\@\xspace}
\newcommand{\consttime}{$\Ovon{1}$\@\xspace}
\newcommand{\linspace}{$\Ovon{N}$\@\xspace}
\newcommand{\slpspace}{$\Ovon{n}$\@\xspace}
\newcommand{\lcastspace}{$\Thetavon{N}$\@\xspace}
\newcommand{\words}{\ensuremath{\Omegavon{\lg\left(N\right)}}}

\newcommand{\leftpart}{\ensuremath{\Sigma^l}}
\newcommand{\rightpart}{\ensuremath{\Sigma^r}}
\newcommand{\blocksymb}{\ensuremath{c}}
\newcommand{\leftsymb}{\ensuremath{c_l}}
\newcommand{\rightsymb}{\ensuremath{c_r}}
\newcommand{\trip}[3]{\ensuremath{\left(#1, #2, #3\right)}}
\newcommand{\parti}[2]{\ensuremath{\left(#1, #2\right)}}
\newcommand{\E}[1]{E\ensuremath{\left\lbrack#1\right\rbrack}}
\newcommand{\Prob}[1]{Pr\ensuremath{\left\lbrack#1\right\rbrack}}

\newcommand{\define}[1]{\emph{\color{blue}{#1}}}

\newcommand{\BComp}{BComp}
\newcommand{\PComp}{PComp}
\newcommand{\TtoG}{TtoG}

\newcommand{\sfspace}[1]{\textsf{#1}\@\xspace}
\newcommand{\fastseqrecomp}{\sfspace{seq\_fast{}}}
\newcommand{\hashrecomp}{\sfspace{seq\_hash{}}}
\newcommand{\naiverecomp}{\sfspace{seq\_naive{}}}
\newcommand{\parallelrecomp}{\sfspace{par\_undircut\_seq{}}}
\newcommand{\parallellprecomp}{\sfspace{par\_undircut\_seq\_prod{}}}
\newcommand{\parallelrndrecomp}{\sfspace{par\_random{}}}
\newcommand{\parallelrnddirrecomp}{\sfspace{par\_random\_dir{}}}
\newcommand{\parallelrndkrecomp}{\sfspace{par\_random10{}}}
\newcommand{\parallelrnddirkrecomp}{\sfspace{par\_random\_dir10{}}}
\newcommand{\parallellsrecomp}{\sfspace{par\_local\_search{}}}
\newcommand{\parallellsgainrecomp}{\sfspace{par\_local\_search\_gain{}}}
\newcommand{\parallelgrrecomp}{\sfspace{par\_greedy{}}}
\newcommand{\simplevec}{\sfspace{SimpleVector}}

\newcommand{\dsdblp}{\sfspace{dblp.xml}}
\newcommand{\dsdna}{\sfspace{dna}}
\newcommand{\dsenglish}{\sfspace{english}}
\newcommand{\dsproteins}{\sfspace{proteins}}
\newcommand{\dssources}{\sfspace{sources}}
\newcommand{\dscoreutils}{\sfspace{coreutils}}
\newcommand{\dsecoli}{\sfspace{Escherichia\_Coli}}
\newcommand{\dseinsteinde}{\sfspace{einstein.de.txt}}
\newcommand{\dseinsteinen}{\sfspace{einstein.en.txt}}
\newcommand{\dsfib}{\sfspace{fib41}}
\newcommand{\dskernel}{\sfspace{kernel}}
\newcommand{\dswl}{\sfspace{world\_leaders}}

\newcommand{\lce}{\sfspace{LCE}}
\newcommand{\lcp}{\sfspace{LCP}}
\newcommand{\lca}{\sfspace{LCA}}
\newcommand{\cfg}{\sfspace{CFG}}
\newcommand{\rmq}{\sfspace{RMQ}}
\newcommand{\slp}{\sfspace{SLP}}
\newcommand{\rlslp}{\sfspace{RLSLP}}
\newcommand{\lz}{\sfspace{LZ77}}
\newcommand{\extract}{\sfspace{Extract}}
\newcommand{\bcomp}{\sfspace{\BComp}}
\newcommand{\pcomp}{\sfspace{\PComp}}
\newcommand{\ttog}{\sfspace{\TtoG}}
\newcommand{\pseq}{\sfspace{PSeq}}
\newcommand{\repair}{\sfspace{RePair}}
\newcommand{\sequitur}{\sfspace{Sequitur}}
\newcommand{\esp}{\sfspace{ESP}}
\newcommand{\prezza}{\sfspace{prezza}}
\newcommand{\naive}{\sfspace{naive}}
\newcommand{\lcequery}[1]{\textsf{LCE}\ensuremath{\left(#1\right)}}
\newcommand{\rmqquery}[2]{\textsf{RMQ}\ensuremath{_{#1}\left(#2\right)}}
\newcommand{\ttogt}{\textsf{TtoG}\ensuremath{(T)}\@\xspace}
\newcommand{\freq}[4]{\textsf{Freq}\ensuremath{_{#1}\left(#2,#3,#4\right)}\@\xspace}
\newcommand{\pseqw}[1]{\textsf{PSeq}\ensuremath{\left(#1\right)}}
\newcommand{\partition}[1]{\textsf{part}\ensuremath{\idx{}{#1}}}
\newcommand{\occ}[1]{\textsf{occ}\ensuremath{\left(#1\right)}}
\newcommand{\extr}[3]{\textsf{Extract}\ensuremath{_#1\left(#2,#3\right)}}

\newcommand{\bcompalg}[3]{\textsf{BComp}\ensuremath{\left(#1, #2, #3\right)}}
\newcommand{\pcompalg}[3]{\textsf{PComp}\ensuremath{\left(#1, #2, #3\right)}}

\newcommand{\deriv}[2]{\ensuremath{\texttt{#1}\rightarrow\texttt{#2}}}

\newcommand{\V}{\ensuremath{\mathcal{V}}}
\newcommand{\D}{\ensuremath{\mathcal{D}}}
\newcommand{\Dl}{\ensuremath{\mathcal{D}^l}}
\newcommand{\Dr}{\ensuremath{\mathcal{D}^r}}
\newcommand{\val}[2]{\ensuremath{val_{#1}\left(#2\right)}}


% Specific include commands
\newcommand{\includechapter}[1]{\input{chapters/#1}}
\newcommand{\includepic}[1]{\input{pictures/#1}}
\newcommand{\includeresult}[1]{\input{results/#1}}
\newcommand{\includecompression}[1]{\includeresult{compression/#1}}
\newcommand{\includelcequeries}[1]{\includeresult{lcequeries/#1}}
\newcommand{\includelonglcequeries}[1]{\includeresult{longlcequeries/#1}}
\newcommand{\includememory}[1]{\includeresult{memory/#1}}
\newcommand{\includerandomaccess}[1]{\includeresult{randomaccess/#1}}
\newcommand{\includeruntime}[1]{\includeresult{runtime/#1}}
\newcommand{\includespeedup}[1]{\includeresult{speedup/#1}}
\newcommand{\includesequential}[1]{\includeresult{sequential/#1}}
\newcommand{\includegraphic}[2]{\includegraphics[#2]{pictures/#1}}

\newcommand{\includeresults}[2]{
	\foreach \res in #2{
		\includeresult{#1/\res}
	}
}
